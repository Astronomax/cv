\documentclass[letterpaper,11pt]{article}

\usepackage{latexsym}
\usepackage[empty]{fullpage}
\usepackage{titlesec}
\usepackage{marvosym}
\usepackage[usenames,dvipsnames]{color}
\usepackage{verbatim}
\usepackage{enumitem}
\usepackage{hyperref}
\usepackage{fancyhdr}
\usepackage{xcolor}
\usepackage{setspace}
\usepackage[utf8]{inputenc} % Поддержка UTF-8
\usepackage[T2A]{fontenc}   % Поддержка кириллических шрифтов
\usepackage[russian]{babel} % Русская локализация

% Цветовая схема
\definecolor{primary}{HTML}{2c3e50}    % Темно-синий для заголовков
\definecolor{secondary}{HTML}{3498db}  % Синий для акцентов
\definecolor{gray}{HTML}{7f8c8d}       % Серый для второстепенного текста

\pagestyle{fancy}
\fancyhf{}
\fancyfoot{}
\renewcommand{\headrulewidth}{0pt}
\renewcommand{\footrulewidth}{0pt}

% Уменьшил поля документа
\addtolength{\oddsidemargin}{-0.6in}   % Было -0.5in
\addtolength{\evensidemargin}{-0.6in}  % Было -0.5in
\addtolength{\textwidth}{1.2in}        % Было 1in
\addtolength{\topmargin}{-0.7in}       % Было -0.5in
\addtolength{\textheight}{1.2in}       % Было 1.0in

\urlstyle{same}

\raggedbottom
\raggedright
\setlength{\tabcolsep}{0in}
\setlength{\parindent}{0pt}

% Форматирование секций - уменьшил размер шрифта заголовков
\titleformat{\section}{
  \vspace{-3pt}\scshape\raggedright\large\color{primary}  % Изменено с \Large на \large
}{}{0em}{}

%-------------------------
% Кастомные команды

\newcommand{\resumeItem}[2]{
  \item\small{
    \textbf{\color{primary}#1}{: \color{black}#2 \vspace{-2pt}}
  }
}

\newcommand{\resumeSubheading}[4]{
  \vspace{-3pt}\item  % Уменьшил отступ
    \begin{tabular*}{0.97\textwidth}{l@{\extracolsep{\fill}}r}
      \textbf{\color{primary}#1} & \color{gray}#2 \\
      \textit{\small\color{gray}#3} & \textit{\small\color{gray} #4} \\
    \end{tabular*}\vspace{-6pt}  % Уменьшил отступ
}

\newcommand{\resumeSubheadingSimple}[3]{
  \vspace{-3pt}\item  % Уменьшил отступ
    \begin{tabular*}{0.97\textwidth}{l@{\extracolsep{\fill}}r}
      \textbf{\color{primary}#1} & \color{gray}#2 \\
      \textit{\small\color{gray}#3} & \\
    \end{tabular*}\vspace{-6pt}  % Уменьшил отступ
}

\newcommand{\resumeSubItem}[2]{\resumeItem{#1}{#2}\vspace{-4pt}}

\renewcommand{\labelitemii}{$\color{secondary}\circ$}
\renewcommand{\labelitemi}{$\color{secondary}\vcenter{\hbox{\tiny$\bullet$}}$}

\newcommand{\resumeSubHeadingListStart}{\begin{itemize}[leftmargin=*, label={}]}
\newcommand{\resumeSubHeadingListEnd}{\end{itemize}}
\newcommand{\resumeItemListStart}{\begin{itemize}[leftmargin=*]}
\newcommand{\resumeItemListEnd}{\end{itemize}\vspace{-5pt}}

\newcommand{\link}[2]{\href{#1}{#2}}

%-------------------------------------------
%%%%%%  CV STARTS HERE  %%%%%%%%%%%%%%%%%%%%%%%%%%%%


\begin{document}

%----------HEADING-----------------
\begin{tabular*}{\textwidth}{l@{\extracolsep{\fill}}r}
  \textbf{\LARGE\color{primary} Антон Кузнец} & \href{mailto:fgfgfb93@gmail.com}{fgfgfb93@gmail.com} \\  % Изменено с \Huge на \LARGE
  & Telegram: \href{https://t.me/Astronomax}{@Astronomax} \\
  \link{https://github.com/Astronomax}{github.com/Astronomax} & \\
  \link{https://codeforces.com/profile/Astronomax}{codeforces.com/profile/Astronomax} & \\
\end{tabular*}

%-----------SUMMARY-----------
\vspace{2pt}  % Уменьшил отступ
Основные интересы: алгоритмы, структуры данных, базы данных, оптимизации производительности.

%-----------WORK EXPERIENCE-----------
\section{Опыт работы}
  \resumeSubHeadingListStart
    \resumeSubheading
      {Разработчик ПО}{Авг. 2023 -- Наст. время}
      {СУБД Tarantool (VK digital technologies)}{}
      \begin{itemize}
        \item Улучшения в системе репликации
        \item Улучшения в MVCC
      \end{itemize}
  \resumeSubHeadingListEnd

%--------------EDUCATION--------------
\section{Образование}
  \resumeSubHeadingListStart
    \resumeSubheading
      {Магистр компьютерных наук}{Сент. 2024 -- Наст. время}
      {Санкт-Петербургский государственный университет}{}
    \resumeSubheading
      {Бакалавр компьютерных наук}{Сент. 2020 -- Июль 2024}
      {Санкт-Петербургский государственный университет}{}
    \resumeSubheadingSimple
      {Алгоритмическая параллель A', Tinkoff Generation}{2019-2020}{}
    \resumeSubheadingSimple
      {Алгоритмическая параллель B+, ШОП, Университет Иннополис}{Июль 2019}{}
  \resumeSubHeadingListEnd
  
%-----------PROJECTS-----------------
\section{Проекты}
  \resumeSubHeadingListStart
    \resumeSubItem{\link{https://github.com/Astronomax/vrptw-powerful-route-minimization-heuristic}{Эвристика минимизации маршрутов VRPTW}}
      {Реализация эвристического алгоритма для задачи маршрутизации транспортных средств с временными окнами. \textbf{C, C++}}

    \resumeSubItem{\link{https://github.com/Astronomax/blob-detection-and-tracking}{Библиотека обнаружения и отслеживания blob-объектов}}
      {Высокопроизводительная библиотека компьютерного зрения на C++/Python для обнаружения и отслеживания blob-объектов. \textbf{C++, Python}}

    \resumeSubItem{\link{https://github.com/Astronomax/compilers-supplementary}{Компилятор Lama (курсовая работа по компиляторам)}}
      {Курсовая работа по курсу компиляторов. Реализует компилятор из языка Lama в байт-код и ассемблер, а также рекурсивный интерпретатор байт-кода стековой машины, оба написаны на языке Lama. \textbf{Lama}}
    \resumeSubItem{\link{https://github.com/Astronomax/lamai}{Интерпретатор байт-кода Lama (курсовая работа по виртуальным машинам)}}
      {Итеративный интерпретатор байт-кода стековой машины образовательного языка программирования Lama. \textbf{C}}
    
    \resumeSubItem{\link{https://github.com/Astronomax/graphics-course-practice}{Задания по курсу компьютерной графики}}
      {Набор практических заданий, выполненных для университетского курса компьютерной графики. Реализует различные основные алгоритмы компьютерной графики и техники рендеринга. \textbf{C++, GLSL}}

    \resumeSubItem{\link{https://github.com/Astronomax/raytracing-seminar-practice}{Практические задания по семинару про raytracing}}
      {Набор практических заданий с семинара по трассировке лучей. Реализует основные алгоритмы трассировки лучей для рендеринга 3D-сцен, включая пересечение геометрии, освещение и симуляцию материалов. \textbf{C++}}
  \resumeSubHeadingListEnd

%-----------COMPETITIVE PROGRAMMING-----------
\section{Спортивное программирование}
  \resumeSubHeadingListStart
    \resumeSubheadingSimple
      {ICPC Северо-Запад России (1/4)}{2021}{12 место}
    \resumeSubheadingSimple
      {Индивидуальная олимпиада школьников по информатике и программированию (ИОИП)}{2020}{диплом 1 степени}
    \resumeSubheadingSimple
      {Московская олимпиада школьников по информатике}{2020}{диплом 2 степени}
    \resumeSubheadingSimple
      {Олимпиада СПбГУ по информатике}{2020}{диплом 2 степени}
    \resumeSubheadingSimple
      {Олимпиада Высшая проба по информатике}{2020}{диплом 2 степени}
  \resumeSubHeadingListEnd

%--------PROGRAMMING LANGUAGES------------
\section{Языки программирования}
\begin{tabular*}{\textwidth}{@{\extracolsep{\fill}}ll}
  \textbf{Основные:} C, C++ & \textbf{Знаком с:} Go, Rust, Python, Kotlin, JavaScript, Haskell 
\end{tabular*}

%-------------------------------------------

\end{document}
